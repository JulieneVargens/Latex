\documentclass[12pt,a4paper]{article}
\usepackage[utf8]{inputenc}
\usepackage{amsmath}
\usepackage{amsfonts}
\usepackage{amssymb}
\usepackage{graphicx}
\usepackage[left=3.0cm,right=2cm,top=3.0cm,bottom=2cm]{geometry}
\author{Juliene Vargens}
\begin{document}
\begin{center}
\textbf{Cálculo da deflexão da luz por uma massa pontual }\\
\end{center}

\begin{enumerate}

\paragraph{}
\item Dada a relação
$\phi(b,z)=\frac{-GM}{\sqrt{b^2+z^2}}$ , calcular 
$\bigtriangledown_{\perp}\phi(b,z)$.

\paragraph{}\textbf{Resolução:}
\begin{equation}
\bigtriangledown_{\perp}\phi(b,z)=\frac{\partial\phi}{\partial b}={\frac{GMb}{(b^2+z^2)^\frac{3}{2}}}
\end{equation}

\item Encontre $\hat{\alpha}= \frac{2}{c^2}\displaystyle{\int\limits_{-\infty}^{\infty}\mid\bigtriangledown_{\perp}\phi\mid dz =\frac{4GM}{c^2b}}$
\subparagraph{}\textbf{Resolução:}

\end{enumerate}
Substituindo a equação (1) em $\hat{\alpha}$, tem-se:
\begin{equation}
\hat{\alpha}= \frac{2}{c^2}{\displaystyle{\int\limits_{-\infty}^{\infty}}\mid\\ \frac{GMb}{(b^2+z^2)^\frac{3}{2}}\ \mid dz}= \frac{2}{c^2}{\displaystyle{\int\limits_{-\infty}^{\infty}} \frac{GMb}{(b^2+z^2)^\frac{3}{2}} dz}
\end{equation}
Resolvendo a integral (2) pelo método de subtituição trigonométrica ,fazendo:
\begin{equation}
z=b\tan\theta\ e \ dz=b\sec^2\theta
\end{equation}
Temos para os limites de integração que :
$z\rightarrow\infty\ quando\ \theta=\frac{\pi}{2}\ e \ z\rightarrow-\infty\ quando\ \theta=-\frac{\pi}{2}$\\
\paragraph{} Então, substituindo os novos limites de integração e (3) na equação (2), temos:
\begin{equation}
\frac{2}{c^2}{\displaystyle{\int\limits_{-\frac{\pi}{2}}^{\frac{\pi}{2}}}\frac{GMb}{(b^2+(b\tan\theta)^2)^\frac{3}{2}}b\sec^2\theta d\theta}=\frac{2}{c^2}{\displaystyle{\int\limits_{-\frac{\pi}{2}}^{\frac{\pi}{2}}}\frac{GMb^2\sec^2\theta d\theta}{(b^2(1+\tan^2\theta))^\frac{3}{2}}}
\end{equation}
{Utilizando a relação $1+\tan^2\theta=\sec^2\theta$}, tem-se que:
\begin{equation}
\frac{2}{c^2}{\displaystyle{\int\limits_{-\frac{\pi}{2}}^{\frac{\pi}{2}}}\frac{GMb^2\sec^2\theta d\theta}{(b^2(\sec^2\theta))^\frac{3}{2}}}= \frac{2}{c^2}{\displaystyle{\int\limits_{-\frac{\pi}{2}}^{\frac{\pi}{2}}}\frac{GMb^2\sec^2\theta d\theta}{b^3\sec^3\theta}}
\end{equation}
Simplicando o numerador com o denominador e colocando os termos constantes para fora da integral, ficamos com:
\begin{equation}
 \frac{2GM}{c^2b}{\displaystyle{\int\limits_{-\frac{\pi}{2}}^{\frac{\pi}{2}}}\frac{d\theta}{\sec\theta}}=\frac{2GM}{c^2b}{\displaystyle{\int\limits_{-\frac{\pi}{2}}^{\frac{\pi}{2}}}\cos\theta\ d\theta}=\frac{2GM}{c^2b}[\\sen\theta]\limits_{-\frac{\pi}{2}}^{\frac{\ \pi}{\ 2}}\\
= \frac{2GM}{c^2b}[1-(-1)]
\end{equation}

Logo:

\begin{equation}
\hat{\alpha}= \frac{4GM}{c^2b}
\end{equation}

\begin{flushright}
C.Q.D
\end{flushright}
\paragraph{} $\hat{\alpha}$ representa o ângulo desvio da luz produzido pelo campo gravitacional de uma massa pontual, conforme mostra a figura abaixo: 
\begin{figure}[!htb]
    \centering
    \includegraphics{mp}
    \caption{representação do desvio da luz por uma massa pontual}
    \label{figRotulo}
    \end{figure}

\end{document}